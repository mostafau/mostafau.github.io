
\documentclass[12pt]{amsart}
\usepackage{geometry} % see geometry.pdf on how to lay out the page. There's lots.
\geometry{a4paper} % or letter or a5paper or ... etc
% \geometry{landscape} % rotated page geometry

% See the ``Article customise'' template for come common customisations

\title{}
\author{}
\date{} % delete this line to display the current date

%%% BEGIN DOCUMENT
\begin{document}

\maketitle


\section{Biography}
I am a third year PhD student in the computer science department at Old Dominion University. I passed my breadth exam in November 2012. Currently I am a graduate research assistant in the SwimSys Lab under Dr. Tamer Nadeem. I also have worked as Research Associate Intern in HP Lab, Palo Alto CA for summer 2013. I did my BSc from CSE department of Bangladesh University of Engineeing \& Technology (BUET). I also worked as a Software Engineer for 4 years at KAZ Software and SDSL.

I have already published my work in several well known international conferences and workshops, such as in MCS 2014, SECON 2014, BHI 2014,  BIBM 2013, PerCom 2013, HotMobile 2013, HomeSys 2012, IWCMC 2012. Besides, I have presented my work in the demo/poster session of several conferences like MobiCom 2013,  INFOCOM 2012, HotMobile 2012. My recent work, "A2PSM: Audio Assisted Wi-Fi Power Saving Mechanism for Smart Devices" at HotMobile'13 has received media attention from an UK based computer topic magazine. In my graduate studies, I have been awarded the Outstanding Research Assistant (Fall 2012) in Computer Science Department, Old Dominion University. I have received NSF student travel grant award to present my work in MobiCom'2013 and PerCom 2013 conference. In addition, I was selected for the final round of Student Research Competition(SRC) Presentation at MobiCom 2013. Also I have filled a patent titled "Wireless Software Defined Network".
%Recently, I have filled two Invention disclosure from my .  


\section{Research Projects:}
My research focus is in Mobile Computing and Wireless Network. Following are the list of my research projects.

\subsection{Wireless Network:}


\subsubsection{Audio-WiFi:}  In this project we integrate the mic/speaker of the smart phones as a parallel communication channel with the Wi-Fi. Our idea is to design and develop a novel framework of communication using mic/speaker in order to improve the efficiency of the Wi-Fi network communication for smart devices. 

\subsubsection{Mobile Extension of SDN(meSDN):} Now-a-days large number of mobile devices use numerous apps that access internet through wireless. With such significant amount of traffic growth and variability, it is now necessary to have greater visibility and control over the traffic generated from the client devices, such that we can ensure performance guarantees to multiple types of users on a shared network infrastructure. Therefore, in this project we extend the SDN framework to the client devices to realize services such as WLAN virtualization with end-to-end QoS, and we propose a framework called meSDN. 

\subsubsection{AppVision:} As mobile applications become more dynamic (rapid install/update), diverse and complex (one app generating multiple flow types), the scalability and granularity requirements also challenge the reliability, coverage and cost of the DPI solutions. This project is an effort to give another light to ML as a viable solution by 1) proposing device-crowd-sourcing for ground truth collection, 2) addressing the limitation of supervised ML techniques � false classification of a flow from a new/unseen application into one of known classes � and 3) significantly improving flowtype classification accuracy via a rich feature set that jointly captures time-frequency information.

\subsection{Mobile Computing:}

\subsubsection{EnergySniffer:} In this project, we develop and evaluate the feasibility of using smart phones in monitoring the energy consumption of the home appliances and machines. We call our system EnergySniffer in which it exploits various sensors, such as magnetic sensor, light, microphone, temperature, camera, WiFi, in smart phones to build a multi sensing framework. This framework is used to build a unique fingerprint profile for each individual machine and home appliance.   

\subsubsection{SpyLoc:} In this project, we design, implement and evaluate the SpyLoc localization system. The design goal of SpyLoc is to develop a light weight and high accuracy localization system for off-the-shelf smartphones. SpyLoc leverages both the acoustic interface (microphone/speaker) and the Wi-Fi interface at the kernel-level of smartphones as well as the inertial sensors in smartphones to achieve high localization accuracy. We implement and evaluated the complete SpyLoc using commercial off-the-shelf smartphones.  

\subsubsection{SmartSpaghetti:} An important tool of the Lean management is the "Spaghetti Diagram", which helps to establish the optimum layout for a department or ward based on observations of the distances traveled by patients, staff and/or products (e.g., x-ray
machines). In this project we use accelerometer, gyroscope, and compass sensors to develop an automated tool to create spaghetti diagrams of movements of personnel in a non-intrusive way.





\end{document}